\documentclass{beamer}
\usepackage[T1, T2A]{fontenc}
\usepackage[utf8]{inputenc}	
\usepackage[english,ngerman]{babel} 
% Kyrillisch und Deutsche Zeichensatz
\usepackage{colortbl}
\usepackage{eurosym}
\usepackage{beamerthemesplit}
\begin{document}
\title{Einführung in Docker} 
\author{Jörg Reuter}
\date{\today} 

\frame{\titlepage} 

\frame{\frametitle{Inhaltsverzeichnis}\tableofcontents} 

\section{Einführung} 
\frame{\frametitle{Docker - Alter Wein in neuen Schläuchen?} 
Gib te schon lange:
\begin{itemize}
\item Containe (OpenVZ, Solris Zones, LXC)r
\item chroot jail
\end{itemize}
und hatten alle Ihre Nachteile ...
    
}
\subsection{Warum Docker}
\frame{ 
\begin{itemize}
    \item Docker ist sehr "leichtgewichtig" z.B. 2500 WebServer unter Docker auf einem Rasberry Pi 2: \url{https://blog.docker.com/2015/10/raspberry-pi-dockercon-challenge-winner/})
    \item Docker verwendet modernste Kernel-Feauture wie:
    \begin{itemize}
        \item eigenes Netzwerk
        \item Eigene Gruppen und Namespaces
        \item Eigener Speicher
        \item Eigenes Ressourcenmangment
        \item Extrem einfacher Zugriff auf Container. 
        \item container können als Textdatei weitergeben werden
    \end{itemize}
\end{itemize}    
}


\section{Komponenten von Docker} 
\frame{\frametitle{Docker besteht aus fünf Teilen:}
\begin{itemize}
    \item Docker Client
    \item Docker Images  
    \item Register
    \item Docker Container
    \item Docker Dämon/Server
\end{itemize} 
Aufbau: \url{https://docs.docker.com/engine/installation/images/win_docker_host.svg}
}

\frame{\frametitle{ mit Pausen}
\begin{itemize}
\item  Einf\"uhrungskurs in \LaTeX \pause 
\item  Kurs 2 \pause 
\item  Seminararbeiten und Pr\"asentationen mit \LaTeX \pause 
\item  Die Beamerclass
\end{itemize} 
}

\subsection{Listen II}
\frame{\frametitle{Numerierte Liste}
\begin{enumerate}
\item  Einf\"uhrungskurs in \LaTeX 
\item  Kurs 2
\item  Seminararbeiten und Pr\"asentationen mit \LaTeX 
\item  Die Beamerclass
\end{enumerate}
}
\frame{\frametitle{Numerierte Liste mit Pausen}
\begin{enumerate}
\item  Einf\"uhrungskurs in \LaTeX \pause 
\item  Kurs 2 \pause 
\item  Seminararbeiten und Pr\"asentationen mit \LaTeX \pause 
\item  Die Beamerclass
\end{enumerate}
}

\section{Abschnitt Nr.3} 
\subsection{Tabellen}
\frame{\frametitle{Tabellen}
\begin{tabular}{|c|c|c|}
\hline
\textbf{Zeitpunkt} & \textbf{Kursleiter} & \textbf{Titel} \\
\hline
WS 04/05 & Sascha Frank &  Erste Schritte mit \LaTeX  \\
\hline
SS 05 & Sascha Frank & \LaTeX \ Kursreihe \\
\hline
\end{tabular}}


\frame{\frametitle{Tabellen mit Pause}
\begin{tabular}{c c c}
A & B & C \\ 
\pause 
1 & 2 & 3 \\  
\pause 
A & B & C \\ 
\end{tabular} }


\section{Abschnitt Nr. 4}
\subsection{Bl\"ocke}
\frame{\frametitle{Bl\"ocke}

\begin{block}{Blocktitel}
Blocktext 
\end{block}

\begin{exampleblock}{Blocktitel}
Blocktext 
\end{exampleblock}


\begin{alertblock}{Blocktitel}
Blocktext 
\end{alertblock}
}
\end{document}
